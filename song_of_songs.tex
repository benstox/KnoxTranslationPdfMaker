%TEXT FROM:
%http://www.newadvent.org/bible/1ma001.htm

\documentclass[10pt]{book} % use larger type; default would be 10pt

\usepackage[latin1]{luainputenc}

\usepackage[oldstyle]{libertine}
\usepackage{lettrine}
\usepackage{xcolor}

% dual language parallel text thing
\usepackage{reledmac}
\usepackage{reledpar}

% BLUES
\definecolor{benblue1}{HTML}{2B22C7}

% REDS
\definecolor{benred8}{HTML}{E82C00} %seems good, maybe the best?

% THREE COMMANDS TO CLEAN UP ALL THE ELEDPAR/ELEDMAC STUFF
\newcommand{\StartOfLatin}{
	\begin{pairs}
	\begin{Leftside}
	\firstlinenum{10000}\linenumincrement{10000}\beginnumbering\pstart
}

\newcommand{\StartOfEnglish}{
	\pend\endnumbering
	\end{Leftside}
	\begin{Rightside}

	\firstlinenum{10000}\linenumincrement{10000}\beginnumbering\pstart
}

\newcommand{\EndOfEnglish}{
	\pend\endnumbering
	\end{Rightside}
	\end{pairs}
	\Columns
}

\usepackage{fancyhdr}
\pagestyle{fancy}
\cfoot{\textcolor{benblue1}{\thepage}}
\renewcommand{\headrulewidth}{0pt}

%redefine 'plain' page styel to have blue page numbers too
\fancypagestyle{plain}{
	\chead{}
	\cfoot{\textcolor{benblue1}{\thepage}}

}

\begin{document}

\chapter*{Song of Songs}

\begin{large}\begin{center}\textsc{Chapter i}\end{center}\end{large}
\lettrine[lines=3]{A}{} kiss from those lips!\footnote[1]{It is not certain, here or throughout the book, whether we are dealing with a series of disconnected love-songs, or with a continuous drama. The present rendering has been divided up into paragraphs on the assumption that a kind of dramatic unity is present, though we cannot always be certain who is the speaker. The first three verses are perhaps flattery addressed to king Solomon by the women of his court.} Wine cannot ravish the senses like that embrace, \textcolor{benred8}{2}~nor the fragrance of rare perfumes match it for delight. Thy very name spoken soothes the heart like flow of oil; what wonder the maids should love thee? \textcolor{benred8}{3}~Draw me after thee where thou wilt; see, we hasten after thee, by the very fragrance of those perfumes allured! To his own bower the king has brought me; he is our pride and boast, on his embrace, more ravishing than wine, our thoughts shall linger. They love truly that know thy love.
\textcolor{benred8}{4}~Dark of skin, and yet I have beauty, daughters of Jerusalem.\footnote[2]{vv. 4-6: The speaker seems to be a village girl, newly brought to the palace, and still thinking of her absent lover. \textasciigrave A vineyard\textquotesingle , i.e. a sweetheart, cf. 8.12 below.} Black are the tents they have in Cedar; black are Solomon\textquotesingle s own curtains; then why not I? \textcolor{benred8}{5}~Take no note of this Ethiop colour; it was the sun tanned me, when my own brothers, that had a grudge against me, set me a-watching in the vineyards. I have a vineyard of my own that I have watched but ill. \textcolor{benred8}{6}~Tell me, my true love, where is now thy pasture-ground, where now is thy resting-place under the noon\textquotesingle s heat? Thou wouldst not have me wander to and fro where the flocks graze that are none of thine?
\textcolor{benred8}{7}~Still bewildered, fairest of womankind?\footnote[3]{vv. 7-10: Spoken to her by king Solomon.} Nay, if thou wilt, wander abroad, and follow with the shepherds\textquotesingle  flocks; feed, if thou wilt, those goats of thine beside the shepherds\textquotesingle  encampment. \textcolor{benred8}{8}~My heart\textquotesingle s love, prized above all my horsemen, with Pharao\textquotesingle s wealth of chariots behind them! \textcolor{benred8}{9}~Soft as doves are thy cheeks, thy neck smooth as coral. \textcolor{benred8}{10}~Chains of gold that neck must have, inlaid with silver.
\textcolor{benred8}{11}~Now, while the king sits at his wine, breathes out the spikenard of my thoughts!\footnote[4]{vv. 11-16: \textasciigrave The spikenard of my thoughts\textquotesingle ; in the original, simply \textasciigrave my spikenard\textquotesingle . But it has been suggested that the words are meant to introduce a new access of reverie. The interruption in verse 14 may be either spoken words from Solomon, or an imagined address by the absent lover.} \textcolor{benred8}{12}~Close my love is to my heart as the cluster of myrrh that lodges in my bosom all the night through. \textcolor{benred8}{13}~Close he clings as a tuft of cypress in the vine-clad rocks of Engedi.
\textcolor{benred8}{14}~See how fair is the maid I love! Soft eyes thou hast, like a dove\textquotesingle s eyes.
\textcolor{benred8}{15}~And see how fair is the man I love, how stately! Green grows that bower, thine and mine, \textcolor{benred8}{16}~with its roof of cedars, with a covert of cypress for its walls.
\begin{large}\begin{center}\textsc{Chapter ii}\end{center}\end{large}
\lettrine[lines=2]{C}{ount} me no more than wild rose on the lowland plain, wild lily on the moun-tain slopes.\footnote[1]{vv. 1-6: The village girl appears to be speaking, except in verse 2, which may be attributed to Solomon. \textasciigrave He\textquotesingle  is the country lover in verse 3, Solomon in verse 6.}
\textcolor{benred8}{2}~A lily, matched with these other maidens, a lily among the brambles, she whom I love!
\textcolor{benred8}{3}~An apple-tree in the wild woodland, shade cool to rest under, fruit sweet to the taste, such is he my heart longs for, matched with his fellows.
\textcolor{benred8}{4}~Into his own banqueting-hall the king has brought me, shewn me the blazon of his love. \textcolor{benred8}{5}~Cushioned on flowers, apples heaped high about me, and love-sick all the while! \textcolor{benred8}{6}~His left hand pillows my head; his right hand, even now, ready to embrace me.
\textcolor{benred8}{7}~An oath, maidens of Jerusalem! By the gazelles and the wild fawns I charge you, wake never from her sleep my heart\textquotesingle s love, till wake she will!\footnote[2]{The end of this verse is sometimes taken literally in the Hebrew text, as meaning \textasciigrave do not arouse or excite (the sentiment of) love (in me) until it pleases to awake of its own accord\textquotesingle . But the Latin rendering, which interprets \textasciigrave love\textquotesingle  as \textasciigrave the loved one\textquotesingle  seems far simpler. If it is right, Solomon is the speaker; the village girl has fallen asleep over the banquet.}
\textcolor{benred8}{8}~The voice I love! See where he comes, how he speeds over the mountains, how he spurns the hills!\footnote[3]{v. 8 of this chapter - v. 4 of the next chapter. Since this passage begins and ends with a warning that the sleeper must not be awoken, the pictures recorded in it are evidently those of a dream. In verse 9, the dreamer seems to echo the half-heard utterance of verse 7.} \textcolor{benred8}{9}~Gazelle nor fawn was ever so fleet of foot as my heart\textquotesingle s love. And now he is standing on the other side of this very wall; now he is looking in through each window in turn, peering through every chink. \textcolor{benred8}{10}~I can hear my true love calling to me: Rise up, rise up quickly, dear heart, so gentle, so beautiful, rise up and come with me. \textcolor{benred8}{11}~Winter is over now, the rain has passed by. \textcolor{benred8}{12}~At home, the flowers have begun to blossom; pruning-time has come; we can hear the turtle-dove cooing already, there at home. \textcolor{benred8}{13}~There is green fruit on the fig-trees; the vines in flower are all fragrance. Rouse thee, and come, so beautiful, so well beloved, \textcolor{benred8}{14}~still hiding thyself as a dove hides in cleft rock or crannied wall. Shew me but thy face, let me but hear thy voice, that voice sweet as thy face is fair.
\textcolor{benred8}{15}~How was it they sang? Catch me the fox, the little fox there, thieving among the vineyards; vineyards of ours, all a-blossoming!\footnote[4]{This is usually thought to be the text of some country song; the words \textasciigrave How was it they sang?\textquotesingle  have been inserted above, so as to prepare the reader for this.}
\textcolor{benred8}{16}~All mine, my true love, and I all his; see where he goes out to pasture among the lilies, \textcolor{benred8}{17}~till the day grows cool, and the shadows long. Come back, my heart\textquotesingle s love, swift as gazelle or fawn out on the hills of Bether.
\begin{large}\begin{center}\textsc{Chapter iii}\end{center}\end{large}
\lettrine[lines=2]{I}{n} the night watches, as I lay abed, I searched for my heart\textquotesingle s love, and searched in vain. \textcolor{benred8}{2}~Now to stir abroad, and traverse the city, searching every alley-way and street for him I love so tenderly! But for all my search I could not find him. \textcolor{benred8}{3}~I met the watchmen who go the city rounds, and asked them whether they had seen my love; \textcolor{benred8}{4}~then, when I had scarce left them, I found him, so tenderly loved; and now that he is mine I will never leave him, never let him go, till I have brought him into my own mother\textquotesingle s house, into the room that saw my birth.
\textcolor{benred8}{5}~An oath, maidens of Jerusalem! By the gazelles and the wild fawns I charge you, wake never from her sleep my heart\textquotesingle s love, till wake she will!
\textcolor{benred8}{6}~Who is this that makes her way up by the desert road, erect as a column of smoke, all myrrh and incense, and those sweet scents the perfumer knows?\footnote[1]{It is difficult to see how this verse fits into its surroundings. Some would translate \textasciigrave What is it that makes its way up\ \ldots\ \textquotesingle  and treat verse 7 as the answer; but the analogy of 8.5 suggests that the reference is somehow to the heroine of the poem.}
\textcolor{benred8}{7}~See now the bed whereon king Solomon lies, with sixty warriors to guard him, none braver in Israel;\footnote[2]{vv. 7-11: These verses are plainly an interlude, in the form of a song (perhaps chanted by the women of Jerusalem) in honour of king Solomon\textquotesingle s state litter.} \textcolor{benred8}{8}~swordsmen all, well trained for battle, and each with his sword girt about him, against the perils of the night! \textcolor{benred8}{9}~A litter king Solomon will have, of Lebanon wood; \textcolor{benred8}{10}~a golden frame it must have, on silver props, with cushions of purple; within are pictured tales of love, for your pleasure, maidens of Jerusalem.\footnote[3]{Literally, \textasciigrave Within, it was inlaid with love, on account of (in the Hebrew text, from) the daughters of Jerusalem\textquotesingle .} \textcolor{benred8}{11}~Come out, maidens of Sion, and see king Solomon wearing the crown that was his mother\textquotesingle s gift to him on his day of triumph, the day of his betrothal.
\begin{large}\begin{center}\textsc{Chapter iv}\end{center}\end{large}
\lettrine[lines=2]{H}{ow} fair thou art, my true love, how fair!\footnote[1]{This chapter forms a love-song which has no special reference to any particular situation; they may be understood as words addressed to the village girl by her lover, and heard either literally or in the imagination.} Eyes soft as dove\textquotesingle s eyes, half-seen behind thy veil; hair that clusters thick as the flocks of goats, when they come home from the Galaad hills; \textcolor{benred8}{2}~teeth white as ewes fresh from the washing, well matched as the twin lambs that follow them; barren is none. \textcolor{benred8}{3}~Thy lips a line of scarlet, guardians of that sweet utterance; thy cheeks shew through their veil rosy as a halved pomegranate. \textcolor{benred8}{4}~Thy neck rising proudly, nobly adorned, like David\textquotesingle s embattled tower, hung about with a thousand shields, panoply of the brave; \textcolor{benred8}{5}~graceful thy breasts as two fawns that feed among the lilies.
\textcolor{benred8}{6}~Till the day grows cool, and the shadows long, myrrh-scented mountain and incense-breathing hill shall be my home.
\textcolor{benred8}{7}~Fair in every part, my true love, no fault in all thy fashioning! \textcolor{benred8}{8}~Venture forth from Lebanon, and come to me, my bride, my queen that shall be! Leave Amana behind thee, Sanir and Hermon heights, where the lairs of lions are, where the leopards roam the hills.\footnote[2]{\textasciigrave My queen that shall be\textquotesingle ; literally, \textasciigrave thou shalt be crowned\textquotesingle . The Hebrew text has simply \textasciigrave Look down\textquotesingle , or perhaps, \textasciigrave Make thy way down\textquotesingle . It is difficult to see why the various heights of the Lebanon range should be mentioned here; unless, indeed, we may suppose that the house called \textasciigrave the Forest of Lebanon\textquotesingle  (III Kg. 7.2 and elsewhere) had its different parts or rooms named after these peaks.}
\textcolor{benred8}{9}~What a wound thou hast made, my bride, my true love, what a wound thou hast made in this heart of mine! And all with one glance of an eye, all with one ringlet straying on thy neck! \textcolor{benred8}{10}~Sweet, sweet are thy caresses, my bride, my true love; wine cannot ravish the senses like that embrace, nor any spices match the perfume that breathes from thee. \textcolor{benred8}{11}~Sweet are thy lips, my bride, as honey dripping from its comb; honey-sweet thy tongue, and soft as milk; the perfume of thy garments is very incense. \textcolor{benred8}{12}~My bride, my true love, a close garden; hedged all about, a spring shut in and sealed! What wealth of grace is here! \textcolor{benred8}{13}~Well-ordered rows of pomegranates, tree of cypress and tuft of nard; \textcolor{benred8}{14}~no lack there whether of spikenard or saffron, of calamus, cinnamon, or incense-tree,\footnote[3]{\textasciigrave Incense-tree\textquotesingle ; the Latin version here transliterates, \textasciigrave trees of Lebanon\textquotesingle , instead of translating the second noun.} of myrrh, aloes or any rarest perfume. \textcolor{benred8}{15}~A stream bordered with garden; water so fresh never came tumbling down from Lebanon.
\textcolor{benred8}{16}~North wind, awake; wind of the south, awake and come; blow through this garden of mine, and set its fragrance all astir.
\begin{large}\begin{center}\textsc{Chapter v}\end{center}\end{large}
\lettrine[lines=2]{I}{nto} his garden, then, let my true love come, and taste his fruit.\footnote[1]{vv. 1-7: The first of these verses may describe a reunion which presents itself to the imagination of the village girl as she falls asleep; the remainder are evidently a dream, which repeats, with variations, the dream of 3.1-3.} The garden gained, my bride, my heart\textquotesingle s love; myrrh and spices of mine all reaped; the honey eaten in its comb, the wine drunk and the milk, that were kept for me! Eat your fill, lovers; drink, sweethearts, and drink deep!
\textcolor{benred8}{2}~I lie asleep; but oh, my heart is wakeful! A knock on the door, and then my true love\textquotesingle s voice: Let me in, my true love, so gentle, my bride, so pure! See, how bedewed is this head of mine, how the night rains have drenched my hair! \textcolor{benred8}{3}~Ah, but my shift, I have laid it by: how can I put it on again? My feet I washed but now; shall I soil them with the dust? \textcolor{benred8}{4}~Then my true love thrust his hand through the lattice, and I trembled inwardly at his touch. \textcolor{benred8}{5}~I rose up to let him in; but my hands dripped ever with myrrh; still with the choicest myrrh my fingers were slippery, \textcolor{benred8}{6}~as I caught the latch. When I opened, my true love was gone; he had passed me by. How my heart had melted at the sound of his voice! And now I searched for him in vain; there was no answer when I called out to him. \textcolor{benred8}{7}~As they went the city rounds, the watchmen fell in with me, that guard the walls; beat me, and left me wounded, and took away my cloak. \textcolor{benred8}{8}~I charge you, maidens of Jerusalem, fall you in with the man I long for, give him this news of me, that I pine away with love.\footnote[2]{vv. 8-17: These verses, with the first two of the following chapter, form a dialogue in which the village girl, now awake, satisfies the curiosity of her companions about her lover\textquotesingle s appearance, but puts them off with vague guesses as to his whereabouts.}
\textcolor{benred8}{9}~Nay, but tell us, fairest of women, how shall we know this sweetheart of thine from another\textquotesingle s? Why is he loved beyond all else, that thou art so urgent with us?
\textcolor{benred8}{10}~My sweetheart? Among ten thousand you shall know him; so white is the colour of his fashioning, and so red. \textcolor{benred8}{11}~His head dazzles like the purest gold; the hair on it lies close as the high palm-branches, raven hair. \textcolor{benred8}{12}~His eyes are gentle as doves by the brook-side, only these are bathed in milk, eyes full of repose.\footnote[3]{\textasciigrave Eyes full of repose\textquotesingle ; we can only make guesses at the meaning of the Hebrew phrase, \textasciigrave reposing upon fullness\textquotesingle , which the Latin version renders \textasciigrave residing by the floods\textquotesingle .} \textcolor{benred8}{13}~Cheeks trim as a spice-bed of the perfumer\textquotesingle s own tending; drench lilies in the finest myrrh, and you shall know the fragrance of his lips. \textcolor{benred8}{14}~Hands well rounded; gold set with jacynth is not workmanship so delicate; body of ivory, and veins of sapphire blue; \textcolor{benred8}{15}~legs straight as marble columns, that stand in sockets of gold. Erect his stature as Lebanon itself, noble as Lebanon cedar. \textcolor{benred8}{16}~Oh, that sweet utterance! Nothing of him but awakes desire. Such is my true love, maidens of Jerusalem; such is the companion I have lost.
\textcolor{benred8}{17}~But where went he, fairest of women, this true love of thine? Tell us what haunts he loves, and we will come with thee to search for him.
\begin{large}\begin{center}\textsc{Chapter vi}\end{center}\end{large}
\lettrine[lines=2]{W}{here} should he be, my true love, but among the spices; where but in his garden, gathering the lilies? \textcolor{benred8}{2}~All mine, my true love, and I all his; ever he would choose the lilies for his pasture-ground.\footnote[1]{Verses 1, 2 evidently continue the thought of the preceding chapter.}
\textcolor{benred8}{3}~Fair thou art and graceful, my heart\textquotesingle s love; for beauty, Jerusalem itself is not thy match; yet no embattled array so awes men\textquotesingle s hearts.\footnote[2]{vv. 3-9: The allusions in verses 4-6 (cf. 4.1-3 above) suggest that the village girl is being addressed; but this time, it would seem, by king Solomon (cf. vv. 7, 8). That he should hit upon the same terms of comparison is perhaps a stroke of deliberate art.} \textcolor{benred8}{4}~Turn thy eyes away, that so unman me! Hair dazzling as the goats have, when they come flocking home from the Galaad hills; \textcolor{benred8}{5}~teeth white as ewes fresh from the washing, well matched as the twin lambs that follow them; barren is none; \textcolor{benred8}{6}~thy cheeks shew through their veil rosy as skin of pomegranate! \textcolor{benred8}{7}~What are three score of queens, and eighty concubines, and maids about them past all counting? \textcolor{benred8}{8}~One there is beyond compare; for me, none so gentle, none so pure! Only once her mother travailed; she would have no darling but this. Maid was none that saw her but called her blessed; queen was none, nor concubine, but spoke in her praise. \textcolor{benred8}{9}~Who is this, whose coming shews like the dawn of day? No moon so fair, no sun so majestic, no embattled array so awes men\textquotesingle s hearts.
\textcolor{benred8}{10}~But when I betook me to the fruit garden, to find apples in the hollows, to see if vine had flowered there, and pomegranate had budded, \textcolor{benred8}{11}~all unawares, my heart misgave me~\ldots\  beside the chariots of Aminadab.\footnote[3]{vv. 10, 11: There is no clue to the speaker; naturally we assume that it is still king Solomon. A comparison of the words used with verse 1 above and 7.8 below suggests that it was his intention to make the village girl his bride. At this point, the text seems to play us false; the statement (both in the Hebrew and in the Septuagint Greek), \textasciigrave My soul made me into the chariots of Aminadab\textquotesingle  (or, of my noble people), is one which gives no tolerable sense. It is probably implied that the speaker swooned away, but the exact meaning of the verse is irrecoverable, and it is not even certain that there may not be a serious gap in the text of the poem.}
\textcolor{benred8}{12}~Come back, maid of Sulam, come back; let us feast our eyes on thee. Maid of Sulam, come back, come back!\footnote[4]{This verse, in which the world \textasciigrave Sulamite\textquotesingle  occurs for the first time, belongs in its context to the succeeding chapter.}
\begin{large}\begin{center}\textsc{Chapter vii}\end{center}\end{large}
\lettrine[lines=2]{W}{hat} can the woman of Sulam give you to feast your eyes on, if it be not the dance of the Two Camps?\footnote[1]{vv. 1-9: The first sentence is presumably spoken by the Sulamite herself, the rest by Solomon. It is commonly assumed that this woman of Sulam (or Sunam, III Kg. 1.3) is the village girl who was the heroine of the preceding chapters. But this is not stated; and we are free, if we will, to regard her as a new character in the drama; a dancer whose charms, lavishly displayed, distract king Solomon from his former love. At the end of verse 5 the Hebrew text probably means \textasciigrave a king is held captive by thy ringlets\textquotesingle , which confirms the impression that king Solomon is the speaker.}Ah, princely maid, how dainty are the steps of thy sandalled feet! Thighs well shaped as the beads of a necklace, some master-craftsman\textquotesingle s work; \textcolor{benred8}{2}~navel delicately carved as a goblet, that has ever its meed of liquor, belly rounded like a heap of corn amid the lilies. \textcolor{benred8}{3}~Graceful thy breasts are as two fawns of the gazelle. \textcolor{benred8}{4}~Thy neck rising proudly like a tower, but all of ivory; deep, deep thy eyes, like those pools at Hesebon, under Beth-rabbim Gate; thy nose imperious as the keep that frowns on Damascus from the hill-side. \textcolor{benred8}{5}~Thy head erect as Carmel, bright as royal purple the braided ripples of thy hair. \textcolor{benred8}{6}~How graceful thou art, dear maiden, how fair, how dainty! \textcolor{benred8}{7}~Thy stature challenges the palm tree, thy breasts the clustering vine. \textcolor{benred8}{8}~What thought should I have but to reach the tree\textquotesingle s top, and gather its fruit? Breasts generous as the grape, breath sweet as apples, \textcolor{benred8}{9}~mouth soft to my love\textquotesingle s caress\footnote[2]{v. 9: \textasciigrave To my love\textquotesingle s caress\textquotesingle ; in the original, the phrase is \textasciigrave to him whom I love\textquotesingle , but this introduces utter confusion into the passage, and the change of a single vowel-point gives us \textasciigrave my caresses\textquotesingle  as in 1.1 and elsewhere.} as good wine is soft to the palate, as food to lips and teeth.
\textcolor{benred8}{10}~My true love, I am all his; and who but I the longing of his heart?\footnote[3]{vv. 10-13: The village girl, who has now evidently said good-bye to the court, rejoins her lover.} \textcolor{benred8}{11}~Come with me, my true love; for us the country ways, the cottage roof for shelter. \textcolor{benred8}{12}~Dawn shall find us in the vineyard, looking to see what flowers the vine has, and whether they are growing into fruit; whether the pomegranates are in blossom. And there thou shalt be master of my love. \textcolor{benred8}{13}~The mandrakes, what scent they give! Over the door at home there are fruits of every sort a-drying; I put them by, new and old, for my true love to eat.
\begin{large}\begin{center}\textsc{Chapter viii}\end{center}\end{large}
\lettrine[lines=2]{W}{ould} that thou wert my brother, nursed at my own mother\textquotesingle s breast! Then I could meet thee in the open street and kiss thee, and earn no contemptuous looks.\footnote[1]{vv. 1-14: Although the transitions of thoughts are not always easy to follow, this chapter can be read without difficulty as lovers\textquotesingle  talk, following on the reunion implied in the foregoing chapter. So read, it is curiously graphic, from verse 1, in which the village girl complains of prying eyes, to verse 14, in which her lover complains of being overheard.} \textcolor{benred8}{2}~To my mother\textquotesingle s house I will lead thee, my captive; there thou shalt teach me my lessons, and I will give thee spiced wine to drink, fresh brewed from my pomegranates. \textcolor{benred8}{3}~His left hand pillows my head; his right hand, even now, ready to embrace me! \textcolor{benred8}{4}~An oath, maidens of Jerusalem! Never wake from her sleep my heart\textquotesingle s love, till wake she will!\footnote[2]{vv. 3, 4: The bride, in a drowsy ecstasy, repeats both her own words and Solomon\textquotesingle s words from 2.6, 7.}
\textcolor{benred8}{5}~Who is this that makes her way up by the desert road, all gaily clad, leaning upon the arm of her true love?\footnote[3]{It is not clear whether the first half of this verse is spoken by the bride, or by onlookers; cf. 3.6. The words \textasciigrave all gaily clad\textquotesingle  are in the Septuagint Greek, but not in the Hebrew text. In the second half, the bride speaks, reminding her lover that their trysting-place has been the actual place in which he was born; this is the sense both of the Hebrew text and of the Septuagint Greek, though the Latin version curiously has: \textasciigrave There thy mother was ravished; there she who bore thee was violated\textquotesingle .}When I came and woke thee, it was under the apple-tree, the same where sore distress overtook thy own mother, where she that bore thee had her hour of shame. \textcolor{benred8}{6}~Hold me close to thy heart, close as locket or bracelet fits; not death itself is so strong as love, not the grave itself cruel as love unrequited; the torch that lights it is a blaze of fire. \textcolor{benred8}{7}~Yes, love is a fire no waters avail to quench, no floods to drown; for love, a man will give up all that he has in the world, and think nothing of his loss.
\textcolor{benred8}{8}~A little sister we have, still unripe for the love of man;\footnote[4]{Verses 8, 9 are evidently a countryside song or proverb, which the bride quotes here so as to emphasise (in verse 10) her own faithfulness.} but the day will come when a man will claim her; what cheer shall she have from us then? \textcolor{benred8}{9}~Steadfast as a wall if she be, that wall shall be crowned with silver; yield she as a door yields, we have cedar boards to fasten her. \textcolor{benred8}{10}~And I, I am a wall; impregnable this breast as a fortress; and the man who claimed me found in me a bringer of content.
\textcolor{benred8}{11}~Solomon had a vineyard at Baal-Hamon; and when he gave the care of it to vine-dressers, each of these must pay a thousand silver pieces for the revenue of it. \textcolor{benred8}{12}~A vineyard I have of my own, here at my side; keep thy thousand pieces, Solomon, and let each vine-dresser have his two hundred; not mine to grudge them.\footnote[5]{vv. 11, 12: (Cf. Mt. 21.34.) The Latin version here has translated the proper names as common nouns, which yields no good sense.}
\textcolor{benred8}{13}~Where is thy love of retired garden walks? All the countryside is listening to thee. \textcolor{benred8}{14}~Give me but the word to come away, thy bridegroom, with thee;\footnote[6]{This is the sense of the Hebrew text; in the Latin, verse 14 is addressed by the girl to her lover.} hasten away like gazelle or fawn that spurns the scented hill-side underfoot.


\end{document}






